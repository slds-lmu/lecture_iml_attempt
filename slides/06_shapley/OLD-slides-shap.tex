\documentclass[11pt,compress,t,notes=noshow, aspectratio=169, xcolor=table]{beamer}

\usepackage{../../style/lmu-lecture}
% Defines macros and environments
% This file is included in slides and exercises

% Rarely used fontstyle for R packages, used only in 
% - forests/slides-forests-benchmark.tex
% - exercises/single-exercises/methods_l_1.Rnw
% - slides/cart/attic/slides_extra_trees.Rnw
\newcommand{\pkg}[1]{{\fontseries{b}\selectfont #1}}

% Spacing helpers, used often (mostly in exercises for \dlz)
\newcommand{\lz}{\vspace{0.5cm}} % vertical space (used often in slides)
\newcommand{\dlz}{\vspace{1cm}}  % double vertical space (used often in exercises, never in slides)
\newcommand{\oneliner}[1] % Oneliner for important statements, used e.g. in iml, algods
{\begin{block}{}\begin{center}\begin{Large}#1\end{Large}\end{center}\end{block}}

% Don't know if this is used or needed, remove?
% textcolor that works in mathmode
% https://tex.stackexchange.com/a/261480
% Used e.g. in forests/slides-forests-bagging.tex
% [...] \textcolor{blue}{\tfrac{1}{M}\sum^M_{m} [...]
% \makeatletter
% \renewcommand*{\@textcolor}[3]{%
%   \protect\leavevmode
%   \begingroup
%     \color#1{#2}#3%
%   \endgroup
% }
% \makeatother


\title{Interpretable Machine Learning}
% \author{LMU}
%\institute{\href{https://compstat-lmu.github.io/lecture_iml/}{compstat-lmu.github.io/lecture\_iml}}
\date{}

\begin{document}

% \newcommand{\titlefigure}{figure/sample-dgp-2d.pdf}
\newcommand{\learninggoals}{
\item Understand SHAP as alternative computation of the Shapley value
\item Learn KernelSHAP
\item Learn TreeSHAP}

\lecturechapter{SHAP: SHapley Additive exPlanations}
\lecture{Interpretable Machine Learning}

\begin{vbframe}{SHAP: SHapley Additive exPlanations}
  \begin{itemize}
      \item SHAP is an alternative estimation method for the Shapley value and comes in two variants: KernelSHAP and TreeSHAP.
      \item The model-agnostic KernelSHAP reframes the Shapley value estimation as linear regression model.
      \item TreeSHAP is a model-specific Shapley value estimation algorithm for tree-based models, such as gradient boosted trees. It provides a computational speedup over other estimation methods.
  \end{itemize}
\end{vbframe}


\begin{vbframe}{KernelSHAP}
KernelSHAP treats estimation as a linear model with a specific kernel function to weight the data. TODO: Use some math here

  \begin{itemize}
    \item Sample coalation vectors.
    \item Get prediction for each coalition vector by converting to the original feature space.
    \item Compute the weight for each k with the SHAP kernel $\pi_x(z') = \ldots $.
    \item Fit weighted linear model.
    \item The Shapley values are the coefficients of the linear model.
  \end{itemize}

\framebreak

\begin{itemize}
  \item Coalition vectors can be created by repeated binary draws from $\{0,1\}$.
  \item The input features to the model can be different to the variables for which we compute Shapley values.
  \item For example, for images the features are the pixels, but Shapley values might be computed for bigger patches (= sets of pixels).
  \item A function $h_x$ has to map from the coalition space to the feature space.
  \item For tabular data, $h_x$ is the value of instance $x'$ if $z' = 1$ and a random sample from the data donor if $z' = 0$.
\end{itemize}

\framebreak
  \begin{itemize}
      \item The KernelSHAP formulation shows the link to local surrogate models
      \item In fact KernelSHAP is equivalent to LIME, when LIME uses the same weights.
      \item KernelSHAP can also be linked to other techniques such as DeepLIFT
  \end{itemize}

\end{vbframe}

\begin{vbframe}{TreeSHAP}

  \begin{itemize}
    \item Fast, model-specific alternativ to KernelSHAP.
    \item Uses a different definition of the value function, which takes the conditional expectation: $v = \ldots $
    \item Shapley values that result are to be interpreted in a different way, as the game is different.
    \item for example Null/Dummy features in the original, marginal game, might not be Null features in the new, conditional formulation of the game.


  \framebreak
    \item The computation of $E$ makes use of the tree strutures:
      \begin{itemize}
        \item If $S$ is the full feature set, we follow take the prediction of the respective terminal node
        \item If $S$ is the empty set, w use the average prediction over all terminal nodes.
        \item To get the prediction for feature coalition $S$, all terminal nodes where the path \enquote{agrees} with features values in $S$ are used. For example, when $x_1 = 7$ is in $S$, and we have a node in the tree where the cutoff is $x_1 < 5$ and $x_1 >= 5$ then all terminal nodes on the left ($x_1 < 5$) are removed. And we average over all others.
        \item Problem: Has to be applied for each possible subset $S$.
        \item Solution: Push down all subsets $S$ at the same time, keep track of number of subsets per node, and their respective weights, which is a bit more complex 
      \end{itemize}

  \end{itemize}

\end{vbframe}


\begin{vbframe}{SHAP for images}
\end{vbframe}


\begin{vbframe}{SHAP for text}
\end{vbframe}

\endlecture
\end{document}
