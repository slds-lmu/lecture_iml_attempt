\documentclass{article}
\usepackage{hyperref}       % hyperlinks
\usepackage{graphicx} % Required for inserting images
\usepackage{xcolor}         % colors

%%%%%%%% MY PACKAGES

\usepackage[utf8]{inputenc} % allow utf-8 input
\usepackage[T1]{fontenc}    % use 8-bit T1 fonts
\usepackage{url}            % simple URL typesetting
\usepackage{booktabs}       % professional-quality tables
\usepackage{amsfonts}       % blackboard math symbols
\usepackage{amsmath}
\usepackage{nicefrac}       % compact symbols for 1/2, etc.
\usepackage{microtype}      % microtypography
\usepackage{xcolor}         % colors
%\usepackage[pdftex]{graphicx}
%\usepackage{ngerman} % für deutsche Anführungszeichen
%\usepackage[inline]{enumitem} % inline enumeration
%\usepackage{dsfont}         % Real numbers symbol etc. => wieder entfernt, nehme stattdessen \mathbb{R} aus dem amsfonts package wie von neurips vorgeschlagen
\usepackage{caption}        % For subfigures and captions
\usepackage{subcaption}     % For subfigures and captions
\usepackage{algorithm}
\usepackage{algpseudocode}
% \usepackage[linesnumbered,ruled,vlined]{algorithm2e}        % For algos with referencable line numbers
\usepackage{tikz}
\usetikzlibrary{shapes.geometric}
\usepackage{multirow}
%\usepackage{pdfpages}
\usepackage{amsthm, amssymb}
\usepackage{bm}
\usepackage{pifont}         % For checkmark symbols with \ding{...}
%\usepackage{ulem}
\usepackage{enumitem}   % Custom enumerate labels
\usepackage[textsize=small]{todonotes}


\newtheorem{definition}{Definition}[section] % Defines the definition environment
\newtheorem{proposition}[definition]{Proposition} % Defines the proposition environment, numbering it along with definitions
\newtheorem{example}[definition]{Example}

% Set paragraph indent to zero
\setlength{\parindent}{0pt}

% dependencies: amsmath, amssymb, dsfont
% math spaces
\ifdefined\N
\renewcommand{\N}{\mathds{N}} % N, naturals
\else \newcommand{\N}{\mathds{N}} \fi
\newcommand{\Z}{\mathds{Z}} % Z, integers
\newcommand{\Q}{\mathds{Q}} % Q, rationals
\newcommand{\R}{\mathds{R}} % R, reals
\ifdefined\C
\renewcommand{\C}{\mathds{C}} % C, complex
\else \newcommand{\C}{\mathds{C}} \fi
\newcommand{\continuous}{\mathcal{C}} % C, space of continuous functions
\newcommand{\M}{\mathcal{M}} % machine numbers
\newcommand{\epsm}{\epsilon_m} % maximum error

% counting / finite sets
\newcommand{\setzo}{\{0, 1\}} % set 0, 1
\newcommand{\setmp}{\{-1, +1\}} % set -1, 1
\newcommand{\unitint}{[0, 1]} % unit interval

% basic math stuff
\newcommand{\xt}{\tilde x} % x tilde
\newcommand{\argmin}{\mathop{\mathrm{arg\,min}}} % argmin
\newcommand{\argmax}{\mathop{\mathrm{arg\,max}}} % argmax
\newcommand{\argminlim}{\argmin\limits} % argmin with limits
\newcommand{\argmaxlim}{\argmax\limits} % argmax with limits
\newcommand{\sign}{\operatorname{sign}} % sign, signum
\newcommand{\I}{\mathbb{I}} % I, indicator
\newcommand{\order}{\mathcal{O}} % O, order
\newcommand{\bigO}{\mathcal{O}} % Big-O Landau
\newcommand{\littleo}{{o}} % Little-o Landau
\newcommand{\pd}[2]{\frac{\partial{#1}}{\partial #2}} % partial derivative
\newcommand{\floorlr}[1]{\left\lfloor #1 \right\rfloor} % floor
\newcommand{\ceillr}[1]{\left\lceil #1 \right\rceil} % ceiling
\newcommand{\indep}{\perp \!\!\! \perp} % independence symbol

% sums and products
\newcommand{\sumin}{\sum\limits_{i=1}^n} % summation from i=1 to n
\newcommand{\sumim}{\sum\limits_{i=1}^m} % summation from i=1 to m
\newcommand{\sumjn}{\sum\limits_{j=1}^n} % summation from j=1 to p
\newcommand{\sumjp}{\sum\limits_{j=1}^p} % summation from j=1 to p
\newcommand{\sumik}{\sum\limits_{i=1}^k} % summation from i=1 to k
\newcommand{\sumkg}{\sum\limits_{k=1}^g} % summation from k=1 to g
\newcommand{\sumjg}{\sum\limits_{j=1}^g} % summation from j=1 to g
\newcommand{\summM}{\sum\limits_{m=1}^M} % summation from m=1 to M
\newcommand{\meanin}{\frac{1}{n} \sum\limits_{i=1}^n} % mean from i=1 to n
\newcommand{\meanim}{\frac{1}{m} \sum\limits_{i=1}^m} % mean from i=1 to n
\newcommand{\meankg}{\frac{1}{g} \sum\limits_{k=1}^g} % mean from k=1 to g
\newcommand{\meanmM}{\frac{1}{M} \sum\limits_{m=1}^M} % mean from m=1 to M
\newcommand{\prodin}{\prod\limits_{i=1}^n} % product from i=1 to n
\newcommand{\prodkg}{\prod\limits_{k=1}^g} % product from k=1 to g
\newcommand{\prodjp}{\prod\limits_{j=1}^p} % product from j=1 to p

% linear algebra
\newcommand{\one}{\bm{1}} % 1, unitvector
\newcommand{\zero}{\mathbf{0}} % 0-vector
\newcommand{\id}{\bm{I}} % I, identity
\newcommand{\diag}{\operatorname{diag}} % diag, diagonal
\newcommand{\trace}{\operatorname{tr}} % tr, trace
\newcommand{\spn}{\operatorname{span}} % span
\newcommand{\scp}[2]{\left\langle #1, #2 \right\rangle} % <.,.>, scalarproduct
\newcommand{\mat}[1]{\begin{pmatrix} #1 \end{pmatrix}} % short pmatrix command
\newcommand{\Amat}{\mathbf{A}} % matrix A
\newcommand{\Deltab}{\mathbf{\Delta}} % error term for vectors

% basic probability + stats
\renewcommand{\P}{\mathds{P}} % P, probability
\newcommand{\E}{\mathds{E}} % E, expectation
\newcommand{\var}{\mathsf{Var}} % Var, variance
\newcommand{\cov}{\mathsf{Cov}} % Cov, covariance
\newcommand{\corr}{\mathsf{Corr}} % Corr, correlation
\newcommand{\normal}{\mathcal{N}} % N of the normal distribution
\newcommand{\iid}{\overset{i.i.d}{\sim}} % dist with i.i.d superscript
\newcommand{\distas}[1]{\overset{#1}{\sim}} % ... is distributed as ...

% machine learning
\newcommand{\Xspace}{\mathcal{X}} % X, input space
\newcommand{\Yspace}{\mathcal{Y}} % Y, output space
\newcommand{\Zspace}{\mathcal{Z}} % Z, space of sampled datapoints
\newcommand{\nset}{\{1, \ldots, n\}} % set from 1 to n
\newcommand{\pset}{\{1, \ldots, p\}} % set from 1 to p
\newcommand{\gset}{\{1, \ldots, g\}} % set from 1 to g
\newcommand{\Pxy}{\mathbb{P}_{xy}} % P_xy
\newcommand{\Exy}{\mathbb{E}_{xy}} % E_xy: Expectation over random variables xy
\newcommand{\xv}{\mathbf{x}} % vector x (bold)
\newcommand{\Xv}{\mathbf{X}} % vector X (random variable, bold)
\newcommand{\xtil}{\tilde{\mathbf{x}}} % vector x-tilde (bold)
\newcommand{\yv}{\mathbf{y}} % vector y (bold)
\newcommand{\Yv}{\mathbf{Y}} % vector Y (random variable, bold)
\newcommand{\xy}{(\xv, y)} % observation (x, y)
\newcommand{\xvec}{\left(x_1, \ldots, x_p\right)^\top} % (x1, ..., xp)
\newcommand{\Xmat}{\mathbf{X}} % Design matrix
\newcommand{\allDatasets}{\mathds{D}} % The set of all datasets
\newcommand{\allDatasetsn}{\mathds{D}_n}  % The set of all datasets of size n
\newcommand{\D}{\mathcal{D}} % D, data
\newcommand{\Dn}{\D_n} % D_n, data of size n
\newcommand{\Dtrain}{\mathcal{D}_{\text{train}}} % D_train, training set
\newcommand{\Dtest}{\mathcal{D}_{\text{test}}} % D_test, test set
\newcommand{\xyi}[1][i]{\left(\xv^{(#1)}, y^{(#1)}\right)} % (x^i, y^i), i-th observation
\newcommand{\Dset}{\left( \xyi[1], \ldots, \xyi[n]\right)} % {(x1,y1)), ..., (xn,yn)}, data
\newcommand{\defAllDatasetsn}{(\Xspace \times \Yspace)^n} % Def. of the set of all datasets of size n
\newcommand{\defAllDatasets}{\bigcup_{n \in \N}(\Xspace \times \Yspace)^n} % Def. of the set of all datasets
\newcommand{\xdat}{\left\{ \xv^{(1)}, \ldots, \xv^{(n)}\right\}} % {x1, ..., xn}, input data
\newcommand{\ydat}{\left\{ \yv^{(1)}, \ldots, \yv^{(n)}\right\}} % {y1, ..., yn}, input data
\newcommand{\yvec}{\left(y^{(1)}, \hdots, y^{(n)}\right)^\top} % (y1, ..., yn), vector of outcomes
\newcommand{\greekxi}{\xi} % Greek letter xi
\renewcommand{\xi}[1][i]{\xv^{(#1)}} % x^i, i-th observed value of x
\newcommand{\yi}[1][i]{y^{(#1)}} % y^i, i-th observed value of y
\newcommand{\xivec}{\left(x^{(i)}_1, \ldots, x^{(i)}_p\right)^\top} % (x1^i, ..., xp^i), i-th observation vector
\newcommand{\xj}{\xv_j} % x_j, j-th feature
\newcommand{\xjvec}{\left(x^{(1)}_j, \ldots, x^{(n)}_j\right)^\top} % (x^1_j, ..., x^n_j), j-th feature vector
\newcommand{\phiv}{\mathbf{\phi}} % Basis transformation function phi
\newcommand{\phixi}{\mathbf{\phi}^{(i)}} % Basis transformation of xi: phi^i := phi(xi)

%%%%%% ml - models general
\newcommand{\lamv}{\bm{\lambda}} % lambda vector, hyperconfiguration vector
\newcommand{\Lam}{\bm{\Lambda}}	 % Lambda, space of all hpos
% Inducer / Inducing algorithm
\newcommand{\preimageInducer}{\left(\defAllDatasets\right)\times\Lam} % Set of all datasets times the hyperparameter space
\newcommand{\preimageInducerShort}{\allDatasets\times\Lam} % Set of all datasets times the hyperparameter space
% Inducer / Inducing algorithm
\newcommand{\ind}{\mathcal{I}} % Inducer, inducing algorithm, learning algorithm

% continuous prediction function f
\newcommand{\ftrue}{f_{\text{true}}}  % True underlying function (if a statistical model is assumed)
\newcommand{\ftruex}{\ftrue(\xv)} % True underlying function (if a statistical model is assumed)
\newcommand{\fx}{f(\xv)} % f(x), continuous prediction function
\newcommand{\fdomains}{f: \Xspace \rightarrow \R^g} % f with domain and co-domain
\newcommand{\Hspace}{\mathcal{H}} % hypothesis space where f is from
\newcommand{\fbayes}{f^{\ast}} % Bayes-optimal model
\newcommand{\fxbayes}{f^{\ast}(\xv)} % Bayes-optimal model
\newcommand{\fkx}[1][k]{f_{#1}(\xv)} % f_j(x), discriminant component function
\newcommand{\fh}{\hat{f}} % f hat, estimated prediction function
\newcommand{\fxh}{\fh(\xv)} % fhat(x)
\newcommand{\fxt}{f(\xv ~|~ \thetav)} % f(x | theta)
\newcommand{\fxi}{f\left(\xv^{(i)}\right)} % f(x^(i))
\newcommand{\fxih}{\hat{f}\left(\xv^{(i)}\right)} % f(x^(i))
\newcommand{\fxit}{f\left(\xv^{(i)} ~|~ \thetav\right)} % f(x^(i) | theta)
\newcommand{\fhD}{\fh_{\D}} % fhat_D, estimate of f based on D
\newcommand{\fhDtrain}{\fh_{\Dtrain}} % fhat_Dtrain, estimate of f based on D
\newcommand{\fhDnlam}{\fh_{\Dn, \lamv}} %model learned on Dn with hp lambda
\newcommand{\fhDlam}{\fh_{\D, \lamv}} %model learned on D with hp lambda
\newcommand{\fhDnlams}{\fh_{\Dn, \lamv^\ast}} %model learned on Dn with optimal hp lambda
\newcommand{\fhDlams}{\fh_{\D, \lamv^\ast}} %model learned on D with optimal hp lambda

% discrete prediction function h
\newcommand{\hx}{h(\xv)} % h(x), discrete prediction function
\newcommand{\hh}{\hat{h}} % h hat
\newcommand{\hxh}{\hat{h}(\xv)} % hhat(x)
\newcommand{\hxt}{h(\xv | \thetav)} % h(x | theta)
\newcommand{\hxi}{h\left(\xi\right)} % h(x^(i))
\newcommand{\hxit}{h\left(\xi ~|~ \thetav\right)} % h(x^(i) | theta)
\newcommand{\hbayes}{h^{\ast}} % Bayes-optimal classification model
\newcommand{\hxbayes}{h^{\ast}(\xv)} % Bayes-optimal classification model

% yhat
\newcommand{\yh}{\hat{y}} % yhat for prediction of target
\newcommand{\yih}{\hat{y}^{(i)}} % yhat^(i) for prediction of ith targiet
\newcommand{\resi}{\yi- \yih}

% theta
\newcommand{\thetah}{\hat{\theta}} % theta hat
\newcommand{\thetav}{\bm{\theta}} % theta vector
\newcommand{\thetavh}{\bm{\hat\theta}} % theta vector hat
\newcommand{\thetat}[1][t]{\thetav^{[#1]}} % theta^[t] in optimization
\newcommand{\thetatn}[1][t]{\thetav^{[#1 +1]}} % theta^[t+1] in optimization
\newcommand{\thetahDnlam}{\thetavh_{\Dn, \lamv}} %theta learned on Dn with hp lambda
\newcommand{\thetahDlam}{\thetavh_{\D, \lamv}} %theta learned on D with hp lambda
\newcommand{\mint}{\min_{\thetav \in \Theta}} % min problem theta
\newcommand{\argmint}{\argmin_{\thetav \in \Theta}} % argmin theta

% densities + probabilities
% pdf of x
\newcommand{\pdf}{p} % p
\newcommand{\pdfx}{p(\xv)} % p(x)
\newcommand{\pixt}{\pi(\xv~|~ \thetav)} % pi(x|theta), pdf of x given theta
\newcommand{\pixit}[1][i]{\pi\left(\xi[#1] ~|~ \thetav\right)} % pi(x^i|theta), pdf of x given theta
\newcommand{\pixii}[1][i]{\pi\left(\xi[#1]\right)} % pi(x^i), pdf of i-th x

% pdf of (x, y)
\newcommand{\pdfxy}{p(\xv,y)} % p(x, y)
\newcommand{\pdfxyt}{p(\xv, y ~|~ \thetav)} % p(x, y | theta)
\newcommand{\pdfxyit}{p\left(\xi, \yi ~|~ \thetav\right)} % p(x^(i), y^(i) | theta)

% pdf of x given y
\newcommand{\pdfxyk}[1][k]{p(\xv | y= #1)} % p(x | y = k)
\newcommand{\lpdfxyk}[1][k]{\log p(\xv | y= #1)} % log p(x | y = k)
\newcommand{\pdfxiyk}[1][k]{p\left(\xi | y= #1 \right)} % p(x^i | y = k)

% prior probabilities
\newcommand{\pik}[1][k]{\pi_{#1}} % pi_k, prior
\newcommand{\lpik}[1][k]{\log \pi_{#1}} % log pi_k, log of the prior
\newcommand{\pit}{\pi(\thetav)} % Prior probability of parameter theta

% posterior probabilities
\newcommand{\post}{\P(y = 1 ~|~ \xv)} % P(y = 1 | x), post. prob for y=1
\newcommand{\postk}[1][k]{\P(y = #1 ~|~ \xv)} % P(y = k | y), post. prob for y=k
\newcommand{\pidomains}{\pi: \Xspace \rightarrow \unitint} % pi with domain and co-domain
\newcommand{\pibayes}{\pi^{\ast}} % Bayes-optimal classification model
\newcommand{\pixbayes}{\pi^{\ast}(\xv)} % Bayes-optimal classification model
\newcommand{\pix}{\pi(\xv)} % pi(x), P(y = 1 | x)
\newcommand{\piv}{\bm{\pi}} % pi, bold, as vector
\newcommand{\pikx}[1][k]{\pi_{#1}(\xv)} % pi_k(x), P(y = k | x)
\newcommand{\pikxt}[1][k]{\pi_{#1}(\xv ~|~ \thetav)} % pi_k(x | theta), P(y = k | x, theta)
\newcommand{\pixh}{\hat \pi(\xv)} % pi(x) hat, P(y = 1 | x) hat
\newcommand{\pikxh}[1][k]{\hat \pi_{#1}(\xv)} % pi_k(x) hat, P(y = k | x) hat
\newcommand{\pixih}{\hat \pi(\xi)} % pi(x^(i)) with hat
\newcommand{\pikxih}[1][k]{\hat \pi_{#1}(\xi)} % pi_k(x^(i)) with hat
\newcommand{\pdfygxt}{p(y ~|~\xv, \thetav)} % p(y | x, theta)
\newcommand{\pdfyigxit}{p\left(\yi ~|~\xi, \thetav\right)} % p(y^i |x^i, theta)
\newcommand{\lpdfygxt}{\log \pdfygxt } % log p(y | x, theta)
\newcommand{\lpdfyigxit}{\log \pdfyigxit} % log p(y^i |x^i, theta)

% probababilistic
\newcommand{\bayesrulek}[1][k]{\frac{\P(\xv | y= #1) \P(y= #1)}{\P(\xv)}} % Bayes rule
\newcommand{\muk}{\bm{\mu_k}} % mean vector of class-k Gaussian (discr analysis)

% residual and margin
\newcommand{\eps}{\epsilon} % residual, stochastic
\newcommand{\epsv}{\bm{\epsilon}} % residual, stochastic, as vector
\newcommand{\epsi}{\epsilon^{(i)}} % epsilon^i, residual, stochastic
\newcommand{\epsh}{\hat{\epsilon}} % residual, estimated
\newcommand{\epsvh}{\hat{\epsv}} % residual, estimated, vector
\newcommand{\yf}{y \fx} % y f(x), margin
\newcommand{\yfi}{\yi \fxi} % y^i f(x^i), margin
\newcommand{\Sigmah}{\hat \Sigma} % estimated covariance matrix
\newcommand{\Sigmahj}{\hat \Sigma_j} % estimated covariance matrix for the j-th class

% ml - loss, risk, likelihood
\newcommand{\Lyf}{L\left(y, f\right)} % L(y, f), loss function
\newcommand{\Lypi}{L\left(y, \pi\right)} % L(y, pi), loss function
\newcommand{\Lxy}{L\left(y, \fx\right)} % L(y, f(x)), loss function
\newcommand{\Lxyi}{L\left(\yi, \fxi\right)} % loss of observation
\newcommand{\Lxyt}{L\left(y, \fxt\right)} % loss with f parameterized
\newcommand{\Lxyit}{L\left(\yi, \fxit\right)} % loss of observation with f parameterized
\newcommand{\Lxym}{L\left(\yi, f\left(\bm{\tilde{x}}^{(i)} ~|~ \thetav\right)\right)} % loss of observation with f parameterized
\newcommand{\Lpixy}{L\left(y, \pix\right)} % loss in classification
\newcommand{\Lpiy}{L\left(y, \pi\right)} % loss in classification
\newcommand{\Lpiv}{L\left(y, \piv\right)} % loss in classification
\newcommand{\Lpixyi}{L\left(\yi, \pixii\right)} % loss of observation in classification
\newcommand{\Lpixyt}{L\left(y, \pixt\right)} % loss with pi parameterized
\newcommand{\Lpixyit}{L\left(\yi, \pixit\right)} % loss of observation with pi parameterized
\newcommand{\Lhy}{L\left(y, h\right)} % L(y, h), loss function on discrete classes
\newcommand{\Lhxy}{L\left(y, \hx\right)} % L(y, h(x)), loss function on discrete classes
\newcommand{\Lr}{L\left(r\right)} % L(r), loss defined on residual (reg) / margin (classif)
\newcommand{\lone}{|y - \fx|} % L1 loss
\newcommand{\ltwo}{\left(y - \fx\right)^2} % L2 loss
\newcommand{\lbernoullimp}{\ln(1 + \exp(-y \cdot \fx))} % Bernoulli loss for -1, +1 encoding
\newcommand{\lbernoullizo}{- y \cdot \fx + \log(1 + \exp(\fx))} % Bernoulli loss for 0, 1 encoding
\newcommand{\lcrossent}{- y \log \left(\pix\right) - (1 - y) \log \left(1 - \pix\right)} % cross-entropy loss
\newcommand{\lbrier}{\left(\pix - y \right)^2} % Brier score
\newcommand{\risk}{\mathcal{R}} % R, risk
\newcommand{\riskbayes}{\mathcal{R}^\ast}
\newcommand{\riskf}{\risk(f)} % R(f), risk
\newcommand{\riskdef}{\E_{y|\xv}\left(\Lxy \right)} % risk def (expected loss)
\newcommand{\riskt}{\mathcal{R}(\thetav)} % R(theta), risk
\newcommand{\riske}{\mathcal{R}_{\text{emp}}} % R_emp, empirical risk w/o factor 1 / n
\newcommand{\riskeb}{\bar{\mathcal{R}}_{\text{emp}}} % R_emp, empirical risk w/ factor 1 / n
\newcommand{\riskef}{\riske(f)} % R_emp(f)
\newcommand{\risket}{\mathcal{R}_{\text{emp}}(\thetav)} % R_emp(theta)
\newcommand{\riskr}{\mathcal{R}_{\text{reg}}} % R_reg, regularized risk
\newcommand{\riskrt}{\mathcal{R}_{\text{reg}}(\thetav)} % R_reg(theta)
\newcommand{\riskrf}{\riskr(f)} % R_reg(f)
\newcommand{\riskrth}{\hat{\mathcal{R}}_{\text{reg}}(\thetav)} % hat R_reg(theta)
\newcommand{\risketh}{\hat{\mathcal{R}}_{\text{emp}}(\thetav)} % hat R_emp(theta)
\newcommand{\LL}{\mathcal{L}} % L, likelihood
\newcommand{\LLt}{\mathcal{L}(\thetav)} % L(theta), likelihood
\newcommand{\LLtx}{\mathcal{L}(\thetav | \xv)} % L(theta|x), likelihood
\newcommand{\logl}{\ell} % l, log-likelihood
\newcommand{\loglt}{\logl(\thetav)} % l(theta), log-likelihood
\newcommand{\logltx}{\logl(\thetav | \xv)} % l(theta|x), log-likelihood
\newcommand{\errtrain}{\text{err}_{\text{train}}} % training error
\newcommand{\errtest}{\text{err}_{\text{test}}} % test error
\newcommand{\errexp}{\overline{\text{err}_{\text{test}}}} % avg training error

% lm
\newcommand{\thx}{\thetav^\top \xv} % linear model
\newcommand{\olsest}{(\Xmat^\top \Xmat)^{-1} \Xmat^\top \yv} % OLS estimator in LM


\begin{document}

\newtheorem{theorem}{Theorem}
\newtheorem{proof_sketch}{Proof}
\newtheorem{proofsketch}{Proof (Step-by-Step)}

\textit{TO DO: all the proofs there are very brief or partially false, see the solution to the homework sheet for a correct and complete solution for all axioms}

\section*{Set-based Shapley Values for Privilege Score Contributions}

\textbf{Setup.} 
$k \hat{=} |P|$. Let the Shapley value for $j \in P$ be
\[
\phi_j \;=\;
\sum_{S \subseteq P \setminus \{j\}}
\frac{|S|!\,\bigl(k - |S| -1\bigr)!}{k!}
\;
\bigl[v(S \cup \{j\}) - v(S)\bigr].
\]

\begin{theorem}[Efficiency]
Let $\{\phi_j\}_{j \in P}$ be the Shapley values induced by $v$. Then
\[
\sum_{j=1}^k \phi_j
\;=\;
v(P).
\]
\end{theorem}

\begin{proof_sketch}
The Shapley value for player $j \in P$ is 
\[
\phi_j 
\;=\;
\sum_{S \subseteq P \setminus \{j\}}
\frac{|S|!\,\bigl(k - |S| - 1\bigr)!}{k!}
\;\Bigl[\,v\bigl(S \cup \{j\}\bigr) \;-\; v(S)\Bigr].
\]
Hence
\[
\sum_{j=1}^k \phi_j
\;=\;
\sum_{j=1}^k
\sum_{S \subseteq P \setminus \{j\}}
\frac{|S|!\,(k - |S| - 1)!}{k!}
\;\bigl[v(S \cup \{j\}) \;-\; v(S)\bigr].
\]
Re-index over nonempty subsets $T \subseteq P$ by setting $T = S \cup \{j\}$. This shows each $v(T)$ with $T\neq \varnothing$ arises exactly $|T|$ times in the sum (once for each $j \in T$), but for subsets $T \neq P$, there is also a corresponding negative occurrence that cancels out the positive one. Concretely:
\[
\bigl[v(T) - v(T \setminus \{i\})\bigr]
\quad \text{and} \quad
\bigl[v(T \cup \{i'\}) - v(T)\bigr]
\]
yield $+\,v(T)$ and $-\,v(T)$, respectively, for some $i,i'\notin T$. 
All intermediate $T \neq P$ vanish in pairs, leaving only the net term $v(P) - v(\varnothing)$. Because $v(\varnothing)=0$, the total sum equals $v(P)$. 
\[
\sum_{j=1}^k \phi_j \;=\; v(P).
\]
\end{proof_sketch}

\begin{proofsketch}
\noindent \textbf{Step 1: Split the sum into pieces.}
\[
\sum_{j=1}^k \phi_j 
\;=\;
\sum_{j=1}^k 
\;\sum_{\,S \subseteq P\setminus\{j\}}
\;\frac{|S|!\,\bigl(k - |S| -1\bigr)!}{k!}
\;\Bigl[\,v\bigl(S \cup \{j\}\bigr) - v(S)\Bigr].
\]
Each pair $(j,S)$ contributes a difference $v(S \cup \{j\}) - v(S)$. Let $T = S \cup \{j\}$.

\medskip
\noindent
\textbf{Step 2: Identify how $v(P)$ appears exactly once overall.}

\begin{itemize}
    \item Whenever $T = P$, we must have $j \in P$ and $S = P \setminus \{j\}$. There are exactly $k$ such pairs \((j,S)\) because $j$ can be any of the $k$ elements in $P$.
    \item For each such pair, $\lvert S\rvert = k-1$, so the coefficient becomes
    \[
       \frac{(k-1)!\,(k - (k-1) - 1)!}{k!}
       \;=\;
       \frac{(k-1)!\,(0)!}{k!}
       \;=\;
       \frac{1}{k}.
    \]
    \item Hence each of the $k$ occurrences of $v(P)$ is multiplied by $\tfrac{1}{k}$, summing to $\bigl(\frac{1}{k}\times k\bigr)=1$. Thus \emph{overall}, $v(P)$ is counted once in total. 
\end{itemize}

\medskip
\noindent
\textbf{Step 3: Show that all $v(T)$ with $T \neq P$ cancel.}

For each nonempty $T \subset P$:
% \begin{itemize}
%     \item \textit{Positive occurrence.} If $T = S \cup \{j\}$ for some $j \in T$, then $v(T)$ appears with a $+$ sign in the term $v(S \cup \{j\}) - v(S)$. 
%     \item \textit{Negative occurrence.} If \(T \subset P\) is not the full set, there is at least one element \(j' \notin T\). Hence, we can form the larger set \(T \cup \{j'\}\). In the Shapley sum, this larger set contributes a term
% \[
% v\bigl(T \cup \{j'\}\bigr) \;-\; v(T),
% \]
% which includes precisely the \textit{negative occurrence} of \(v(T)\).  

%     \item Check the multinomial coefficient: Consider a fixed nonempty $T$. Its size is $|T|$. In the inner sum, $T$ arises precisely once for the unique $j \in T$ and $S = T \setminus \{j\}$. The corresponding coefficient is
% \[
% \frac{|T\setminus\{j\}|!\,\bigl(k - |T\setminus\{j\}| -1\bigr)!}{k!}
% \;=\;
% \frac{(|T|-1)!\,\bigl(k - (|T|-1) -1\bigr)!}{k!}
% \;=\;
% \frac{(|T|-1)!\,(k - |T|)!}{k!}.
% \]
% Multiplying by $|T|$ (because there are $|T|$ ways to pick $j$ in $T$ as shown in Step 1) yields
% \[
% |T| \;\cdot\; \frac{(|T|-1)!\,(k - |T|)!}{k!}
% \;=\;
% \frac{|T|! \,(k - |T|)!}{k!}
% \;=\;
% 1.
% \]
% Thus each $v(T)$ with $T\neq \varnothing$ appears with a total net weight of $+1$ (when summing across all $j\in T$).
%     \item Every nonempty \(T \neq P\) appears positively in some difference \(v(T) - v(\cdot)\) and also appears negatively in a difference \(v(\cdot) - v(T)\) as part of a larger set. These two occurrences nullify each other, ensuring that \(v(T)\) does not remain in the final sum.
% \end{itemize}
\begin{itemize}
    \item \textbf{Positive occurrence.} 
    If $T = S \cup \{j\}$ for some $j \in T$, then $v(T)$ appears \emph{positively} in the difference 
    \[
      v(S \cup \{j\}) \;-\; v(S).
    \]

    \item \textbf{Negative occurrence.} 
    If \(T \subset P\) is not the full set, then there exists at least one element \(j' \notin T\). We can thus form the larger set \(T \cup \{j'\}\). In the Shapley sum, this larger set contributes
    \[
      v\bigl(T \cup \{j'\}\bigr) \;-\; v(T),
    \]
    which contains \(-\,v(T)\) as the \emph{negative} appearance of \(v(T)\).

    \item \textbf{Multinomial coefficient.} 
    Fix a nonempty $T$. Let $|T|$ denote its size. Inside the sum, $T$ appears once for each $j \in T$ with $S = T \setminus \{j\}$. The coefficient in front of $v(T)$ for such a pair $(j,S)$ is
    \[
      \frac{|T\setminus\{j\}|!\,\bigl(k - |T\setminus\{j\}| -1\bigr)!}{k!}
      \;=\;
      \frac{(|T|-1)!\,(k - |T|)!}{k!}.
    \]
    Since there are $|T|$ possible ways to choose $j\in T$, multiplying by $|T|$ yields
    \textit{TO DO The following calculation is wrong, this fraction yields $\frac{1}{\binom{k}{|T|}}$, for exactly the reason shown in the proof}
    \[
      |T| 
      \;\cdot\; 
      \frac{(|T|-1)!\,(k - |T|)!}{k!}
      \;=\;
      \frac{|T|!\,(k - |T|)!}{k!}
      \;=\;
      \frac{k!}{k!}
      \;=\;
      1.
    \]
    Where the identity $|T|!\,(k - |T|)! = k!$ holds because to permute $k$ distinct elements, 
we can imagine first choosing which $|T|$ are in one group and ordering them 
(in $|T|!$ ways), then ordering the remaining $k - |T|$ (in $(k - |T|)!$ ways). 
Overall, this accounts for all $k!$ permutations. 
    Therefore, each nonempty $T$ receives a total \emph{positive} weight of exactly $+1$ when summing over all $j \in T$.

    \item \textbf{Cancellation.} 
    \textit{TO DO: The following is completely handwaivy, it is totally unclear why the weights are the same, so that these terms really cancel out to 0}
    Every nonempty \(T \neq P\) also appears \emph{negatively} as part of a larger set’s difference, ensuring one positive and one negative occurrence of $v(T)$. Consequently, these contributions cancel each other out, so \(v(T)\) does not remain in the final sum unless $T=P$.
\end{itemize}
\medskip
\noindent
\textbf{Step 4: Conclusion.}

Since $v(\varnothing) = 0$ does not contribute and $v(P)$ remains once in total, we get
\[
\sum_{j=1}^k \phi_j 
\;=\;
\underbrace{v(P)}_{\text{one net positive}} 
\;+\; 
\underbrace{\sum_{T \neq P} \bigl[v(T)\ \text{terms cancel}\bigr]}_{0} 
\;=\;
v(P).
\]
\qedhere
\end{proofsketch}

% \begin{proofsketch}
% \noindent \textbf{Step 1: Expand the total Shapley sum.}

% \[
% \sum_{j=1}^k \phi_j
% \;=\;
% \sum_{j=1}^k
% \;\sum_{\,S \subseteq P \setminus \{j\}}
% \frac{|S|!\,(k-|S|-1)!}{k!}
% \;\Bigl[v\bigl(S \cup \{j\}\bigr) - v(S)\Bigr].
% \]

% \noindent \textbf{Step 2: Re-index to track each $v(T)$ term.}

% Each nonempty subset \(T \subseteq P\) appears exactly \(\lvert T\rvert\) times positively in the expanded total Shapley sum from Step 1. More precisely, for each \(j \in T\), there is a unique set \(S = T \setminus \{j\}\) such that \(T = S \cup \{j\}\). Consequently, each \(v(T)\) appears exactly once in the difference \(v(S \cup \{j\}) - v(S)\) if and only if \(S = T \setminus \{j\}\).

% \textbf{Example:} 
% Let $P = \{1,2,3\}$ and focus on $T = \{1,3\}$.  
% In the Shapley sum, we form terms $v(S \cup \{j\}) - v(S)$ for each $j \in P$ and each $S \subseteq P \setminus \{j\}$. 
% We list all such $(j,S)$ pairs:

% \begin{itemize}
%   \item \textbf{Case $j=1$}, so $S \subseteq \{2,3\}$:
%     \begin{itemize}
%       \item $S = \varnothing$: $S \cup \{1\} = \{1\} \implies v(\{1\}) - v(\varnothing).$
%       \item $S = \{2\}$: $S \cup \{1\} = \{1,2\} \implies v(\{1,2\}) - v(\{2\}).$
%       \item \textbf{$S = \{3\}$:} $S \cup \{1\} = \{1,3\} \implies \textbf{v(\{1,3\}) - v(\{3\})}.$
%       \item $S = \{2,3\}$: $S \cup \{1\} = \{1,2,3\} \implies v(\{1,2,3\}) - v(\{2,3\}).$
%     \end{itemize}

%   \item \textbf{Case $j=2$}, so $S \subseteq \{1,3\}$:
%     \begin{itemize}
%       \item $S = \varnothing$: $S \cup \{2\} = \{2\} \implies v(\{2\}) - v(\varnothing).$
%       \item $S = \{1\}$: $S \cup \{2\} = \{1,2\} \implies v(\{1,2\}) - v(\{1\}).$
%       \item $S = \{3\}$: $S \cup \{2\} = \{2,3\} \implies v(\{2,3\}) - v(\{3\}).$
%       \item $S = \{1,3\}$: $S \cup \{2\} = \{1,2,3\} \implies v(\{1,2,3\}) - v(\{1,3\}).$
%     \end{itemize}

%   \item \textbf{Case $j=3$}, so $S \subseteq \{1,2\}$:
%     \begin{itemize}
%       \item $S = \varnothing$: $S \cup \{3\} = \{3\} \implies v(\{3\}) - v(\varnothing).$
%       \item \textbf{$S = \{1\}$:} $S \cup \{3\} = \{1,3\} \implies \textbf{v(\{1,3\}) - v(\{1\})}.$
%       \item $S = \{2\}$: $S \cup \{3\} = \{2,3\} \implies v(\{2,3\}) - v(\{2\}).$
%       \item $S = \{1,2\}$: $S \cup \{3\} = \{1,2,3\} \implies v(\{1,2,3\}) - v(\{1,2\}).$
%     \end{itemize}
% \end{itemize}

% \noindent
% \textbf{Observation:} $T = \{1,3\}$ occurs \emph{only} in $|T| = 2$ cases, i.e. for the pairs
% \[
% (j,S) = (1,\{3\}) \quad \text{and} \quad (j,S) = (3,\{1\}),
% \]
% matching the principle: $T$ appears once for each $j\in T$, with $S = T \setminus\{j\}$.  
% No other $(j,S)$ pair produces $T=\{1,3\}$, so $T$ cannot arise for $j=2$ or for different $S$.



% \noindent \textbf{Step 3: Check the multinomial coefficient contribution.}

% Consider a fixed nonempty $T$. Its size is $|T|$. In the inner sum, $T$ arises precisely once for the unique $j \in T$ and $S = T \setminus \{j\}$. The corresponding coefficient is
% \[
% \frac{|T\setminus\{j\}|!\,\bigl(k - |T\setminus\{j\}| -1\bigr)!}{k!}
% \;=\;
% \frac{(|T|-1)!\,\bigl(k - (|T|-1) -1\bigr)!}{k!}
% \;=\;
% \frac{(|T|-1)!\,(k - |T|)!}{k!}.
% \]
% Multiplying by $|T|$ (because there are $|T|$ ways to pick $j$ in $T$ as shown in Step 1) yields
% \[
% |T| \;\cdot\; \frac{(|T|-1)!\,(k - |T|)!}{k!}
% \;=\;
% \frac{|T|! \,(k - |T|)!}{k!}
% \;=\;
% 1.
% \]
% Thus each $v(T)$ with $T\neq \varnothing$ appears with a total net weight of $+1$ (when summing across all $j\in T$).

% \noindent \textbf{Step 4: Telescoping.}

% \begin{itemize}
%     \item $v(\varnothing)$ never appears positively, because $\varnothing$ cannot be formed by adding some $j$.
%     \item $v(P)$ never appears negatively, because $P$ cannot be “further extended.”
%     \item All intermediate $v(T)$ terms with $T \neq \varnothing,P$ appear exactly once positively and once negatively, and hence cancel out except for $v(P) - v(\varnothing)$.
% \end{itemize}

% \noindent \textbf{Step 5: Conclude.}

% Since $v(\varnothing) = 0$, the only remaining term is $v(P)$. Hence
% \[
% \sum_{j=1}^k \phi_j \;=\; v(P).
% \]
% \qedhere
% \end{proofsketch}


\section*{Order-Based Shapley Values for Privilege Score Contributions}

\paragraph{Setup.}
Let $P = \{1,\dots,k\}$ be the set of $k$ players.  
We aim to allocate $v(P)$ among the $k$ players (arrows) in a principled way.

\medskip

\paragraph{Permutation-Based Definition of Shapley Values.}
A \emph{permutation} of $P$ is any ordering $\pi = (\pi(1), \pi(2), \dots, \pi(k))$.  
Let $\mathrm{Pred}_\pi(j)$ be the set of players that appear \emph{before} $j$ in the permutation $\pi$.  
Then the \textbf{Shapley value} of player $j$ is
\[
\phi_j 
\;=\;
\frac{1}{k!}\;
\sum_{\pi \in \mathfrak{S}_P}
\Bigl[v\!\bigl(\mathrm{Pred}_\pi(j) \cup \{\,j\}\bigr)
\;-\;
v\!\bigl(\mathrm{Pred}_\pi(j)\bigr)\Bigr],
\]
where $\mathfrak{S}_P$ is the set of all $k!$ permutations of $P$.  
In words, we look at the “marginal contribution” of $j$ each time it arrives in a permutation (given whichever players arrived before it), and then average over all permutations.

\begin{theorem}[Efficiency]
Let $\{\phi_j\}_{j \in P}$ be the order-based Shapley values induced by $v$. Then
\[
\sum_{j=1}^k \phi_j
\;=\;
v(P).
\]
\end{theorem}

\begin{proofsketch}
\noindent
\textbf{Step 1: Sum the Shapley values over all players.}

\[
\sum_{j=1}^k \phi_j 
\;=\;
\sum_{j=1}^k
\frac{1}{k!}
\sum_{\pi \in \mathfrak{S}_P}
\Bigl[v\bigl(\mathrm{Pred}_\pi(j)\cup\{j\}\bigr)
- 
v\bigl(\mathrm{Pred}_\pi(j)\bigr)\Bigr].
\]
Swap the sums:
\[
=
\frac{1}{k!}
\sum_{\pi \in \mathfrak{S}_P}
\sum_{j=1}^k
\Bigl[v\bigl(\mathrm{Pred}_\pi(j)\cup\{j\}\bigr)
- 
v\bigl(\mathrm{Pred}_\pi(j)\bigr)\Bigr].
\]

\medskip
\noindent
\textbf{Step 2: Telescoping within each permutation.}

Fix a particular permutation $\pi$.  List its elements in order:
\[
\bigl(\pi(1), \pi(2), \dots, \pi(k)\bigr).
\]
Within this permutation, the inner sum
\(\sum_{j=1}^k
\bigl[v(\mathrm{Pred}_\pi(j)\cup\{j\}) - v(\mathrm{Pred}_\pi(j))\bigr]\)
can be viewed as a chain of marginal contributions:
\[
v\bigl(\{\pi(1)\}\bigr) - v(\varnothing)
\;+\;
v\bigl(\{\pi(1), \pi(2)\}\bigr) - v\bigl(\{\pi(1)\}\bigr)
\;+\;\dots\;+\;
v\bigl(\{\pi(1),\ldots,\pi(k)\}\bigr) - v\bigl(\{\pi(1),\ldots,\pi(k-1)\}\bigr).
\]
All intermediate terms telescope, leaving exactly
\[
v\bigl(\{\pi(1),\ldots,\pi(k)\}\bigr)
\;-\;
v(\varnothing)
\;=\;
v(P)
\;-\;
0
\;=\;
v(P).
\]

\medskip
\noindent
\textbf{Step 3: Average over all permutations.}

Since every permutation $\pi$ yields exactly $v(P)$ in the telescoped sum, we have
\[
\sum_{j=1}^k \phi_j
\;=\;
\frac{1}{k!}
\sum_{\pi \in \mathfrak{S}_P} 
v(P)
\;=\;
\frac{1}{k!} \times \bigl(k! \cdot v(P)\bigr)
\;=\;
v(P).
\]
Hence
\[
\sum_{j=1}^k \phi_j 
\;=\;
v(P).
\]
\end{proofsketch}

\end{document}
